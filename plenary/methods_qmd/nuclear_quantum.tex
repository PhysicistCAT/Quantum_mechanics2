\begin{frame}{Methods Including Nuclear Quantum Effects}
    \textbf{Path Integral Molecular Dynamics (PIMD)}
    \begin{itemize}
        \item Centroid Molecular Dynamics (CMD)
        \item Ring Polymer Molecular Dynamics (RPMD)
    \end{itemize}
    \vspace{1em}

    \textbf{Wave Packet Propagation}
\end{frame}
\begin{frame}{RPMD for Real-Time Quantum Correlation Functions}

\begin{itemize}
    \item \textbf{The Problem:} The exact calculation of quantum real-time correlation functions is still considered "a very difficult problem".
    
    \item \textbf{General Approach:} A number of methods have been proposed to include short-time quantum mechanical effects in classical molecular dynamics simulations. These methods typically combine an exact treatment of the quantum Boltzmann operator with an approximate treatment of the real-time evolution based on classical mechanics.
    
    \item \textbf{The Proposed Method (RPMD):} We propose an approximate method, based on path integral (Parrinello-Rahman) molecular dynamics, for calculating \textbf{Kubo-transformed real-time correlation functions} involving position-dependent operators.
    
    \item \textbf{Core Idea:} Exploit the isomorphism between the path integral representation of the quantum mechanical partition function and the classical partition function of a fictitious ring polymer.
\end{itemize}
\end{frame}

\begin{frame}{Methodology: Kubo Transform and Dynamics}

\begin{itemize}
    \item \textbf{Object of Interest:} The method focuses on the Kubo-transformed correlation function, $\tilde{C}_{AB}(t)$.
    \begin{itemize}
        \item This function is appealing because it has the same symmetries as a classical correlation function.
    \end{itemize}

    \item \textbf{Initial Condition ($t=0$):} The $t \to 0$ limit of $\tilde{C}_{AB}(t)$ coincides with a purely classical phase space average $\langle A_n(x) B_n(x) \rangle_n$ calculated for the fictitious ring polymer system.
\end{frame}
	\begin{frame}
    \item \textbf{Time Evolution:} To extend this result to times $t > 0$, the method uses the classical dynamics generated by the ring polymer Hamiltonian $H_n(p,x)$.
    \begin{itemize}
        \item The resulting correlation function, $\langle A(0)B(t) \rangle_n$, is calculated via integration over initial phase space variables evolved using the classical equations of motion.
    \end{itemize}
    
    \item \textbf{Symmetry Guarantee:} The classical ring-polymer correlation function $\langle A(0)B(t) \rangle_n$ maintains the crucial symmetries of $\tilde{C}_{AB}(t)$, including the quantum mechanical detailed balance condition.
\end{itemize}
\end{frame}

\begin{frame}{Key Analytical Results and Advantages}

\begin{itemize}
    \item \textbf{RPMD is Exact at $t=0$:} The resulting correlation functions coincide with the exact quantum mechanical result in the limit as $t \to 0$ (for sufficiently large $n$).
    
    \item \textbf{The Harmonic Limit:} RPMD gives the exact quantum mechanical result $\tilde{C}_{AB}(t)$ for all times $t$ in a harmonic potential, provided that one or both operators ($A(\hat{x})$ or $B(\hat{x})$) are linear functions of position.
    \begin{itemize}
        \item For the position autocorrelation function $\tilde{C}_{xx}(t)$, the classical limit ($n=1$ polymer bead) is already exact in the harmonic case.
    \end{itemize}
    
    \item \textbf{Consistent Improvement:} The RPMD correlation functions are consistently better than those given by purely classical molecular dynamics.
    \begin{itemize}
        \item This improvement is \textbf{most apparent in the low-temperature regime} where quantum statistics are critically important.
    \end{itemize}
    
    \item \textbf{Computational Efficiency:} RPMD requires less computational work compared to the related centroid molecular dynamics method, yet its results are only marginally less accurate for tested problems.
\end{itemize}
\end{frame}

\begin{frame}{Limitations and Summary}

\begin{itemize}
    \item \textbf{Short-Time Focus:} The method is designed primarily to give an accurate approximation to the quantum mechanical correlation function for times on the order of the thermal time ($\beta\hbar$).
    
    \item \textbf{Missing Quantum Dynamics:} The neglect of quantum phase information in the subsequent dynamics is clearly undesirable.
    
    \item \textbf{Failure for Long-Time Coherences:} The method misses long-time quantum coherence effects that arise in simple one-dimensional anharmonic systems (e.g., the quartic oscillator).
    \begin{itemize}
        \item This failure is expected, as the method does not contain the phase information needed to capture long-time quantum oscillations.
        \item The method performs well only in situations where quantum effects in the dynamics are comparatively unimportant.
    \end{itemize}
    
    \item \textbf{Summary:} RPMD successfully includes short-time quantum effects stemming from the Boltzmann operator, providing a consistent improvement over classical MD, especially at lower temperatures.
    
    \item \textbf{Future Work:} It will be interesting to test RPMD in condensed phase applications, such as calculating the infrared absorption spectrum of liquid water.
\end{itemize}
\end{frame}

