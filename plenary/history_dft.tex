\begin{frame}{\large I. Origins: The Precursors}
    \framesubtitle{The Density as a Basic Variable}
    \begin{itemize}
        \item The earliest density-based methods trace back to the Thomas, Fermi, and Dirac papers (1927–1930).
        \item \textbf{The Thomas-Fermi-Dirac model} described atoms based purely on the electron density $n(\mathbf{r})$.
        \item The motivation was to find "approximate practical methods" to solve the complicated quantum mechanical equations (Dirac, 1929).
        \item Limitation: The Thomas-Fermi model has severe deficiencies, such as the inability to predict molecular or solid binding.
    \end{itemize}
\end{frame}

% 4. Modern Formalism
\begin{frame}{\large II. Modern Density Functional Formalism (HK)}
    \framesubtitle{The Hohenberg-Kohn Theorems (1964)}
    \begin{itemize}
        \item HK Theorem 1: The external potential $V_{\text{ext}}(\mathbf{r})$ (and thus the full Hamiltonian) is uniquely determined by the ground-state electron density $n(\mathbf{r})$.
        \item HK Theorem 2: A universal functional for the energy, $E[n]$, exists, and the correct ground-state density minimizes this Functional.
        \begin{equation*}
            E[n, V_{\text{ext}}] = F[n] + \int n(\mathbf{r}) V_{\text{ext}}(\mathbf{r}) d\mathbf{r} \quad (\text{Variational Principle})
        \end{equation*}
        \item This formal grounding eliminated the need for the complicated $3N$-dimensional wave function $\Psi(\mathbf{r}_1, \ldots, \mathbf{r}_N)$.
    \end{itemize}
\end{frame}

% 5. The KS Scheme
\begin{frame}{\large III. The Kohn-Sham Scheme and Approximations}
    \framesubtitle{The Practical Method (1965)}
    \begin{itemize}
        \item The Kohn-Sham (KS) scheme introduced a non-interacting auxiliary system that produces the exact ground-state density $n(\mathbf{r})$ of the real interacting system.
        \item This turns the many-body problem into a set of solvable single-particle equations (Kohn-Sham equations).
        \item The complexity is now contained in the exchange-correlation functional $E_{xc}[n]$.
        \item Key Approximations:
        \begin{itemize}
            \item Local Density Approximation (LDA): Based on the homogeneous electron gas.
            \item Generalized Gradient Approximation (GGA): Introduced dependence on the density gradient, $\nabla n$.
        \end{itemize}
    \end{itemize}
\end{frame}

% 6. The Breakthrough (1990-Present)
\begin{frame}{\large IV. The Rise to Prominence (Post-1990)}
    \begin{itemize}
        \item Widespread application, particularly in chemistry and materials science, grew astonishingly after 1990.
        \item The number of publications citing DFT/DF increased dramatically, marking a huge success story (Fig. 1).
        \item Crucial Developments:
        \begin{itemize}
            \item Car-Parrinello Molecular Dynamics (1985): Combined DFT with MD, enabling efficient study of structures and reactions.
            \item Improved Functionals: Development of GGAs and hybrid functionals (e.g., incorporating Hartree-Fock exact exchange).
        \end{itemize}
        \item A 1991 conference in Menton is cited as a major turning point for acceptance among chemists.
    \end{itemize}
\end{frame}

% 7. Current Successes
\begin{frame}{\large V. Current Status and Applications}
    \framesubtitle{A Standard Tool in Modern Science}
    \begin{itemize}
        \item DFT is now well-established in condensed matter physics and has a significant presence in theoretical chemistry.
        \item It is particularly valuable for calculating the total energy ($E$) and energy surfaces $E(\mathbf{R}_i)$, which are essential for determining ground-state structures and chemical reaction paths.
        \item DFT allows calculations of complex systems in biochemistry and materials science that were "far beyond expectations" in 1990.
    \end{itemize}
\end{frame}

% 8. Future and Challenges
\begin{frame}{\large VI. Challenges and Future Outlook}
    \framesubtitle{Quo Vadis? (Whither Goest Thou?)}
    \begin{itemize}
        \item Despite its success, prominent practitioners have raised concerns about the future ("best of times and the worst of times").
        \item \textbf{The Central Problem:} The lack of a systematic way to improve the \textbf{Exchange-Correlation functional ($E_{xc}$)}.
        \item \textbf{Current Challenges:}
        \begin{itemize}
            \item Accurately describing \textbf{Dispersion (van der Waals) interactions}.
            \item Treating \textbf{"Strongly correlated" systems}.
            \item The proliferation of hundreds of approximate functionals makes choosing the "best" one ambiguous and risks a "semi-empirical" view of the theory.
        \end{itemize}
        \item The field is continually exploring extensions, such as combining DFT with Density Matrix Functional Theory (DMFT) or Quantum Monte Carlo (QMC).
    \end{itemize}
\end{frame}

% 9. Conclusion
\begin{frame}{\large VII. Conclusion}
    \begin{itemize}
        \item DFT is a huge success story, having moved from theoretical origins to a mainstream computational tool due to the work of Hohenberg, Kohn, and Sham and the subsequent development of reliable functional approximations.
        \item Its widespread acceptance came after 1990, driven by its efficiency and ability to handle large systems, a goal anticipated by Dirac in 1929.
        \item The future of DFT rests on the continuing development of accurate, universal, and physically sound exchange-correlation functionals.
    \end{itemize}
\end{frame}