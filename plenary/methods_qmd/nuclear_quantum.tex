\begin{frame}{Methods Including Nuclear Quantum Effects}
    \textbf{Path Integral Molecular Dynamics (PIMD)}
    \begin{itemize}
        \item Centroid Molecular Dynamics (CMD)
        \item Ring Polymer Molecular Dynamics (RPMD)
    \end{itemize}
    \vspace{1em}

    \textbf{Wave Packet Propagation}
\end{frame}

\subsection{PIMD}

%---------------------PIMD------------------------------------------

\begin{frame}{PIMD (Path Integral Molecular Dynamics)}
	\begin{block}{Paper Title}
	Study of an F center in molten KCl
	\end{block}
\end{frame}

\begin{frame}
  \frametitle{The Problem: A Quantum Particle in a Classical World}
  
  Simulating systems with both quantum and classical particles presents a major challenge. \pause
  
  \begin{itemize}
    \item Consider an electron (a quantum particle) dissolved in a bath of classical ions, like molten salt. \pause
    
    \item The electron does not have a fixed position; its quantum nature means it is "fuzzy" and delocalized. \pause
    
    \item How can we combine the quantum statistical mechanics of the electron with the classical statistical physics of the ions in a single simulation?
  \end{itemize}
\end{frame}

\begin{frame}
  \frametitle{The Path Integral Solution: A Quantum-Classical Isomorphism}
  
  The method uses Richard Feynman's path integral formulation to map the quantum problem onto an equivalent classical problem. \pause
  
  \begin{block}{The Isomorphism}
    A single quantum particle is formally proven to be equivalent to a classical \textbf{ring polymer} (or "necklace") made of P beads.
  \end{block} \pause
  
  \begin{itemize}
    \item Each bead in the necklace is connected to its two neighbors by \textbf{harmonic springs}. This part of the model represents the quantum kinetic energy. \pause
    
    \item Every bead also interacts with the external potential created by the classical particles. \pause
    
    \item This mapping becomes exact as the number of beads (P) approaches infinity. In practice, a finite but sufficiently large P is used.
  \end{itemize}
\end{frame}

\begin{frame}
  \frametitle{Simulating the Necklace: PIMD in Action}
  
  Once the quantum particle is represented as a classical polymer, the entire system can be simulated using standard \textbf{Molecular Dynamics (MD)}. \pause
  
  \begin{alertblock}{Physical Interpretation}
    \begin{itemize}
      \item The \textbf{spatial spread} of the polymer beads directly represents the quantum "fuzziness" or delocalization of the particle. \pause
      \begin{itemize}
          \item A compact, collapsed necklace means the particle is highly localized (like an F center).
          \item A spread-out, large necklace means the particle is delocalized. \pause
      \end{itemize}
      \item This allows us to calculate quantum properties like kinetic energy and study phenomena like electron localization, all within a classical simulation framework.
    \end{itemize}
  \end{alertblock}
  
\end{frame}

%-----------------CMD---------------------------------------------

\begin{frame}{CMD (centroid molecular dynamics)}
	\begin{block}{Paper Title}
	The formulation of quantum statistical mechanics based on the Feynman path
centroid density. II. Dynamical properties
	\end{block}
\end{frame}

\begin{frame}
  \frametitle{Beyond Static Properties: The Challenge of Quantum Dynamics}
  
  The Path Integral formalism (PIMD) is excellent for calculating equilibrium quantum properties by mapping a particle to a classical ring polymer. \pause
  
  \begin{itemize}
    \item However, the "dynamics" used in a PIMD simulation are fictitious; they are simply a tool to sample configurations. \pause
    
    \item This leaves a critical question open: How can we calculate \textbf{real-time dynamical properties}, such as time correlation functions $\langle A(t)B(0)\rangle$? \pause
    
    \item We need a way to extract real physical evolution from the path integral framework.
  \end{itemize}
\end{frame}

\begin{frame}
  \frametitle{Centroid Molecular Dynamics (CMD): The Core Idea}
  
  The CMD method proposes that the most important dynamical information is captured by a special variable: the \textbf{path centroid}. \pause
  
  \begin{itemize}
    \item The centroid is the center of mass (average position) of all the beads in the path integral polymer. It is considered the most direct classical-like variable in the quantum system. \pause
    
    \item \textbf{The Central Approximation:} The full, complex quantum dynamics of a particle can be effectively approximated by the \textbf{classical evolution of its centroid}. \pause
    
    \item However, the centroid does not move on the classical potential. It evolves on a \textbf{quantum potential of mean force} ($V_c$). \pause
    
    \item This "centroid potential" is an effective potential averaged over all the quantum fluctuations of the polymer beads, and it implicitly includes quantum effects like zero-point energy and tunneling.
  \end{itemize}
\end{frame}

\begin{frame}
  \frametitle{CMD in Action: A "Quasiclassical" Simulation}
  
  The CMD method provides a practical algorithm for simulating quantum dynamics. \pause
  
  \begin{block}{The Simulation Process}
    A molecular dynamics simulation is performed, but only for the centroid coordinates, which behave like classical particles.
  \end{block} \pause
  
  \begin{itemize}
    \item These centroids evolve according to Newton's equations of motion ($F_c = m \ddot{q}_c$). \pause
    
    \item The force ($F_c$) is not the classical force, but the \textbf{quantum mean force}, derived from the gradient of the centroid potential ($F_c = -\nabla V_c$). \pause
    
    \item The resulting "centroid trajectories" are then used to compute approximate quantum time correlation functions. \pause
  \end{itemize}
  
  \begin{alertblock}{Key Advantage}
    The computational cost of a CMD simulation scales in the same way as a purely classical simulation, making it a powerful tool for studying quantum dynamical effects in large, complex systems.
  \end{alertblock}
\end{frame}