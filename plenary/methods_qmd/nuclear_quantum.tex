\begin{frame}{Methods Including Nuclear Quantum Effects}
    \textbf{Path Integral Molecular Dynamics (PIMD)}
    \begin{itemize}
        \item Centroid Molecular Dynamics (CMD)
        \item Ring Polymer Molecular Dynamics (RPMD)
    \end{itemize}
    \vspace{1em}

    \textbf{Wave Packet Propagation}
\end{frame}

%---------------------PIMD------------------------------------------

\begin{frame}{PIMD (Path Integral Molecular Dynamics)}
	\begin{block}{Paper Title}
	Study of an F center in molten KCl
	\end{block}
\end{frame}

\begin{frame}
  \frametitle{The Problem: A Quantum Particle in a Classical World}
  
  Simulating systems with both quantum and classical particles presents a major challenge. \pause
  
  \begin{itemize}
    \item Consider an electron (a quantum particle) dissolved in a bath of classical ions, like molten salt. \pause
    
    \item The electron does not have a fixed position; its quantum nature means it is "fuzzy" and delocalized. \pause
    
    \item How can we combine the quantum statistical mechanics of the electron with the classical statistical physics of the ions in a single simulation?
  \end{itemize}
\end{frame}

\begin{frame}
  \frametitle{The Path Integral Solution: A Quantum-Classical Isomorphism}
  
  The method uses Richard Feynman's path integral formulation to map the quantum problem onto an equivalent classical problem. \pause
  
  \begin{block}{The Isomorphism}
    A single quantum particle is formally proven to be equivalent to a classical \textbf{ring polymer} (or "necklace") made of P beads.
  \end{block} \pause
  
  \begin{itemize}
    \item Each bead in the necklace is connected to its two neighbors by \textbf{harmonic springs}. This part of the model represents the quantum kinetic energy. \pause
    
    \item Every bead also interacts with the external potential created by the classical particles. \pause
    
    \item This mapping becomes exact as the number of beads (P) approaches infinity. In practice, a finite but sufficiently large P is used.
  \end{itemize}
\end{frame}

\begin{frame}
  \frametitle{Simulating the Necklace: PIMD in Action}
  
  Once the quantum particle is represented as a classical polymer, the entire system can be simulated using standard \textbf{Molecular Dynamics (MD)}. \pause
  
  \begin{alertblock}{Physical Interpretation}
    \begin{itemize}
      \item The \textbf{spatial spread} of the polymer beads directly represents the quantum "fuzziness" or delocalization of the particle. \pause
      \begin{itemize}
          \item A compact, collapsed necklace means the particle is highly localized (like an F center).
          \item A spread-out, large necklace means the particle is delocalized. \pause
      \end{itemize}
      \item This allows us to calculate quantum properties like kinetic energy and study phenomena like electron localization, all within a classical simulation framework.
    \end{itemize}
  \end{alertblock}
  
\end{frame}