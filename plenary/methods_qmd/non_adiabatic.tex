\begin{frame}

    \textbf{Trajectory Surface Hopping (TSH)}
    \vspace{1.5em}
    
    \textbf{Multi-Configuration Time-Dependent Hartree (MCTDH)}
\end{frame}

%------------------TSH-----------------------------------------------

\begin{frame}{TSH}
	\begin{block}{Paper Title}
	Molecular dynamics with electronic transitions
	\end{block}
\end{frame}

\begin{frame}
  \frametitle{The Challenge: When Quantum Electrons Meet Classical Atoms}
  
  Standard Molecular Dynamics is powerful, but it has a fundamental limit: it assumes atoms always move on a \textbf{single potential energy surface}. \pause
  
  \begin{itemize}
    \item This works well for many ground-state processes. \pause
    
    \item But what happens in photochemistry, or during an electron transfer? \pause
    
    \item The electronic state changes, and the forces governing the atoms change drastically. The system effectively jumps to a \textbf{new potential energy surface}. \pause
    
    \item We need a method that can simulate classical atomic trajectories while allowing for quantum electronic transitions.
  \end{itemize}
\end{frame}

\begin{frame}
  \frametitle{The Trajectory Surface Hopping (TSH) Solution}
  
  The TSH method combines classical and quantum mechanics in a self-consistent way. \pause
  
  \begin{block}{The Core Idea}
    A trajectory evolves classically on a single electronic state at any given time, but it has a certain probability of "hopping" to another state.
  \end{block} \pause
  
  Two calculations happen in parallel:
  
  \begin{itemize}
    \item \textbf{Classical Atoms:} The atomic positions evolve according to Newton's equations on the \textit{currently active} potential energy surface. \pause
    
    \item \textbf{Quantum Electrons:} The time-dependent Schrödinger equation is solved for the electrons along this trajectory, yielding the probability of being in \textit{any} of the available electronic states.
  \end{itemize}
\end{frame}

\begin{frame}
  \frametitle{The "Fewest Switches" Algorithm}
  
  The key is deciding when to hop. The TSH method uses a clever probabilistic algorithm. \pause
  
  \begin{itemize}
    \item At each time step, the algorithm calculates the probability of switching from the current state to any other state. \pause
    
    \item This is based on the "fewest switches" criterion: it minimizes the number of hops while ensuring that an ensemble of trajectories correctly reproduces the quantum populations of each state. \pause
    
    \item A random number is used to make a stochastic decision: \textbf{to hop or not to hop}. \pause
    
    \item If a hop occurs, the atoms' velocities are instantly adjusted to conserve total energy, and the trajectory continues on the new potential energy surface.
  \end{itemize}
\end{frame}