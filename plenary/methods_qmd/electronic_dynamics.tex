\begin{frame}{Methods for Electronic Dynamics}{By Propagation Algorithm}
    \textbf{Born-Oppenheimer Molecular Dynamics (BOMD)}
    \vspace{1em}
    
    \textbf{Car-Parrinello Molecular Dynamics (CPMD)}
    \vspace{1em}
    
    \textbf{Ehrenfest Molecular Dynamics}
\end{frame}

\begin{frame}{Methods for Electronic Dynamics}{By Level of Electronic Theory}
    \begin{block}{Density Functional Theory (DFT-MD)}
    \end{block}
    
    \begin{block}{Wavefunction Theory (WFT-MD)}
        \textbf{Hartree-Fock (HF-MD)}
        \vspace{0.5em}
        
        \textbf{Post-Hartree-Fock (MP2-MD, CCSD-MD)}
    \end{block}
    
    \begin{block}{Semi-empirical Methods (SE-MD)}
    \end{block}
\end{frame}

\subsection{Post-Hartree-Fock (MP2-MD, CCSD-MD)}

\begin{frame}{CCSD-MD}
	\begin{block}{Paper Title}
	On the Correlation Problem in Atomic and Molecular Systems. Calculation
of Wavefunction Components in Ursell-Type Expansion Using Quantum
Field Theoretical Methods
	\end{block}
\end{frame}

\begin{frame}{The Problem: Electron Correlation}
    \begin{block}{Main Objective}
        The paper proposes a method to calculate the exact electron wavefunction $|\Psi\rangle$, starting from a single-particle approximation like Hartree-Fock ($|\Phi\rangle$).
    \end{block}
    \pause

    \begin{itemize}
        \item The main challenge is to describe the \textbf{correlation} in the motion of electrons.
        \pause
        \item The difference between the exact energy and the Hartree-Fock energy is the \textbf{correlation energy}.
    \end{itemize}
    \pause

    \begin{alertblock}{The Central Proposal: The Exponential Ansatz}
        The exact wavefunction is expressed as:
        \[
        |\Psi\rangle = e^{\hat{T}} |\Phi\rangle
        \]
        Where $\hat{T}$ is the \textbf{cluster operator}.
    \end{alertblock}
\end{frame}

%------------------------------------------------

\begin{frame}{Field Theory Tools (I)}
    To derive the equations for the $\hat{T}$ operator, the paper uses a robust formalism:
    \pause

    \begin{block}{Second Quantization}
        The Hamiltonian ($\hat{H} = \hat{Z} + \hat{V}$) is written in terms of creation ($\hat{X}_A^+$) and annihilation ($\hat{X}_A$) operators.
    \end{block}
    \pause
    
    \begin{block}{Hole-Particle Formalism}
        \begin{itemize}
            \item The reference state $|\Phi\rangle$ is treated as a "new vacuum."
            \pause
            \item A \textbf{particle} is an electron in an excited (virtual) orbital.
            \pause
            \item A \textbf{hole} is the absence of an electron in an occupied orbital.
            \pause
            \item New operators $\hat{Y}$ are defined to create/annihilate these excitations.
        \end{itemize}
    \end{block}
\end{frame}

%------------------------------------------------

\begin{frame}{Field Theory Tools (II)}
    \begin{block}{Wick's Theorem}
        It's used to simplify products of creation and annihilation operators, breaking them down into normal products and contractions (pairings).
    \end{block}
    \pause
    
    \begin{block}{Feynman-like Diagrams}
        A diagrammatic technique is introduced to systematically visualize and organize the complex algebraic terms that arise from solving the Schrödinger equation.
    \end{block}
\end{frame}

%------------------------------------------------

\begin{frame}{The Cluster Operator $\hat{T}$}
    The operator $\hat{T}$ is a sum of operators that generate all possible particle-hole excitations.
    \[
    \hat{T} = \hat{T}_1 + \hat{T}_2 + \hat{T}_3 + \dots
    \]
    \pause

    \begin{itemize}
        \item<2-> \textbf{$\hat{T}_1$}: Generates single excitations (1 particle, 1 hole).
        \item<3-> \textbf{$\hat{T}_2$}: Generates double excitations (2 particles, 2 holes).
        \item<4-> \textbf{$\hat{T}_j$}: Generates excitations of order *j*.
    \end{itemize}
    \pause

    \begin{exampleblock}<5->{Coupled Cluster Equations}
        By projecting the Schrödinger equation onto the reference state and the excited states, a system of algebraic equations is obtained to determine the coefficients (amplitudes) of the $\hat{T}_j$ operators.
    \end{exampleblock}
\end{frame}

%------------------------------------------------

\begin{frame}{Relation to Configuration Interaction (CI)}
    The Coupled Cluster (CC) method is closely related to CI.
    \pause
    
    \textbf{CI wavefunction (linear expansion):}
    \[
    |\Psi\rangle_{CI} = |\Phi\rangle + \hat{C}_1|\Phi\rangle + \hat{C}_2|\Phi\rangle + \dots
    \]
    \pause
    
    \textbf{CC wavefunction (exponential expansion):}
    \[
    |\Psi\rangle_{CC} = \left(1 + \hat{T} + \frac{1}{2!}\hat{T}^2 + \dots \right) |\Phi\rangle
    \]
    \pause
    
    \begin{block}<4->{Operator Connection}
        \begin{itemize}
            \item $\hat{C}_1 = \hat{T}_1$
            \item $\hat{C}_2 = \hat{T}_2 + \frac{1}{2}\hat{T}_1^2$
            \item $\hat{C}_3 = \hat{T}_3 + \hat{T}_1\hat{T}_2 + \frac{1}{6}\hat{T}_1^3$
            \item[...]
        \end{itemize}
    \end{block}
\end{frame}

\begin{frame}{Fundamental Advantage over Truncated CI}
    \begin{alertblock}{Inclusion of Higher-Order Excitations}
        If we truncate the CC expansion to only $\hat{T}_2$ (the CCSD method), the wavefunction implicitly contains contributions from quadruple excitations through the term $\frac{1}{2}\hat{T}_2^2|\Phi\rangle$.
        \pause
        \vspace{1em}
        
        These are the "disconnected" quadruple excitations, which are often the most important and are omitted in a CI calculation with only doubles (CID). \pause This makes the method \textbf{size-extensive}.
    \end{alertblock}
\end{frame}

\begin{frame}{Applications and Methodology}
    The paper tests the method on two model systems.
    \pause

    \begin{itemize}
        \item \textbf{Models Studied:}
        \begin{itemize}
            \item A simplified description of the N$_2$ molecule.
            \item The $\pi$ system of the benzene molecule.
        \end{itemize}
        \pause
        \bigskip
        \item \textbf{Approximations Compared:}
        \begin{enumerate}
            \item<4-> \textbf{"Full" CI:} The exact reference standard.
            \item<5-> \textbf{CI-D (B):} CI with only double excitations.
            \item<6-> \textbf{Linear-CC (L):} Method with $\hat{T}_2$ and linearized equations.
            \item<7-> \textbf{Nonlinear-CC (N):} Method with $\hat{T}_2$ and nonlinear equations.
        \end{enumerate}
    \end{itemize}
\end{frame}

\begin{frame}{Key Results}
    \begin{block}{Comparison of Correlation Energy}
        \begin{itemize}
            \item CI with Doubles (B) \textbf{underestimates} the correlation energy.
            \pause
            \bigskip
            \item The Linear-CC (L) method \textbf{overestimates} the correlation energy.
            \pause
            \bigskip
            \item The Nonlinear-CC (N) method gives results \textbf{very close and accurate} to the "Full" CI, showing a drastic improvement over CI with Doubles.
        \end{itemize}
    \end{block}
\end{frame}

\begin{frame}{Method Conclusions}
    \begin{itemize}
        \item Jiří Čížek presents a \textbf{non-variational} and \textbf{systematic} method to address the electron correlation problem.
        \pause
        \bigskip
        \item The formulation is based on \textbf{quantum field theory}, unifying concepts to create a computationally tractable framework.
        \pause
        \bigskip
        \item The use of the \textbf{exponential ansatz} ($e^{\hat{T}}$) is key, as it efficiently captures higher-order excitations.
        \pause
        \bigskip
        \item This work laid the foundation for \textbf{Coupled Cluster} methods, which are considered the "gold standard" in quantum chemistry today.
    \end{itemize}
\end{frame}